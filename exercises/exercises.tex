\documentclass[11pt, oneside]{article}

\def\jtitle{Exercises from Silverman and Tate}
\def\jauthor{Jack DeSerrano}
\usepackage[annot]{jack}
\setlength{\parindent}{0pt}



\begin{document}
\maketitle
These are my solutions to the exercises to Joseph Silverman's and John Tate's \underline{Rational Points}\\ \underline{on Elliptic Curves} \cite{rat}.\footnote{See \url{https://link.springer.com/book/10.1007/978-1-4757-4252-7}.} Note that many exercises have been commented out due to the lack of a complete solution. 
\tableofcontents
\section{Geometry and Arithmetic}
\begin{enumerate}
\setlength{\parindent}{0pt}
\item (a) If $P$ and $Q$ are distinct rational points in the $(x,y)$ plane, prove that the line connecting them is a rational line.
\begin{proof}
Let 
$$
P = (a/b, c/d)\quad \textrm{and} \quad Q = (e/f,g/h)
$$
and let $y = \lambda x + \nu$ be the line connecting $P$ and $Q$. Then the slope of this line is rational: 
$$
\lambda = \frac{c/d - g/h}{a/b -e/f}.
$$
The $y$-intercept of this line is also rational:
$$
\nu = c/d-\lambda(a/b).
$$
So the line is rational.
\end{proof}
(b) If $L_1$ and $L_2$ are distinct rational lines in the $(x,y)$ plane, prove that their intersection is a rational point (or empty).
\begin{proof}
Let  
$$
L_1 : y = (a/b)x + c/d \quad\textrm{and}\quad L_2 : (e/f)x + g/h.
$$ 
If $a/b = e/f$, then their intersection is either empty or the whole line, and then the job is done. If not, then the horizontal component of their intersection is rational, as corroborated below.
\begin{align*}
(a/b)x + c/d &= (e/f)x + g/h\\
(a/b - e/f)x &= g/h - c/d\\
x &= \frac{g/h - c/d}{a/b - e/f}
\end{align*}
Since the horizontal component of a point on a rational line is rational, then so is the vertical component. So the intersection is rational.
\end{proof}
%\item Let $C$ be the conic given by the equation
%$$
%F(x,y) = ax^2 + bxy + cy^2 + dx+ ey +f =0,
%$$
%and let $\delta$ be the determinant
%$$
%\det\left ( \begin{array}{ccc} 2a & b & d \\ b & 2c & e \\ d & e & 2f\end{array}\right).
%$$
%(a) Show that if $\delta\neq 0$, then $C$ has no singular points. That is, show that there are no points $(x,y)$ where
%$$
%F(x,y) = \frac{\partial F}{\partial x} (x,y) = \frac{\partial F}{\partial y} (x,y) = 0.
%$$
%\begin{proof} % improve
%Homogenize $F$ such that one attains the equation of the projective conic corresponding to $C$:
%$$
% F^*(x,y,z) = ax^2 + bxy +cy^2 +dxz+eyz+fz^2.t
%$$ 
%Consider a curve of degree $d$ in the projective plane given by $G=0$ for some homogenous polynomial $G[x,y,z]\in\C[x,y,z]$ A point $P$ is singular on this curve if and only if
%$$
%\frac{\partial G}{\partial x} (P) = \frac{\partial G}{\partial y} (P) = \frac{\partial G}{\partial z} (P) = 0.
%$$ 
%A theorem of Euler \cite{hom} guarantees that such a $P$ is on the curve. This notion is applied to $F*$. One notices that the columns of the matrix in question correspond to the coefficients of the partial derivatives of $F^*$. Therefore, the above condition (of the determinant) is necessary for there to exist a singular point on the projective conic defined by $F^*$, and thus on $C$.
%\end{proof}
%(b) Conversely, show that if $\delta = 0$ and $b^2 - 4ac \neq 0$, then there is a unique singular point on $C$. % see https://en.wikipedia.org/wiki/Matrix_representation_of_conic_sections

%\item Let $C$ be the conic given by the equation
%$$
%x^2 - 3xy + 2y^2 - x +1 = 0.
%$$
%(a) Check that $C$ is non-singular.
%\begin{proof}
%A conic defined by the equation 
%$$
%F(x,y) = ax^2 + bxy + cy^2 + dx+ ey +f =0
%$$
%is non-singular if 
%$$
%\det\left ( \begin{array}{ccc} 2a & b & d \\ b & 2c & e \\ d & e & 2f\end{array}\right) \neq 0.
%$$
%Computing the appropriate determinant corroborates that $C$ is non-singular.
%\end{proof}
%(b) Let $L$ be the line $y=\alpha x + \beta$. Suppose that the intersection $L\cap C$ contains the point $P_0 = (x_0,y_0)$. Assuming that $L\cap C$ consists of two distinct points, find the second point of $L\cap C$ in terms of $\alpha$, $\beta$, $x_0$, $y_0$.
\setcounter{enumi}{3}
\item Find all primitive integral right triangles whose hypotenuse has length less than $30$.
\begin{proof}
Choose two coprime positive integers $m > n$ with opposite parities. One attains a primitive triple by 
$$
(m^2 - n^2, 2mn, m^2 + n^2).
$$
Any primitive triple is of this form. One uses this formula to attain all of the primitive integral triples with hypotenuse less than $30$:
\begin{align*}
(3,4,5), \quad(5,12,13), \quad(15, 8, 17), \quad(21, 20, 29), \quad(7, 24, 25).
\end{align*}
\end{proof}
\item Find all of the rational points on the circle 
$$
x^2 + y^2 = 2
$$
by projecting from the point $(1,1)$ onto an appropriate rational line. 
\begin{proof}
Project from $(1,1)$ onto the line $y=-x$. With this projection and $t$ a real number, one obtains the following equation for the line:
$$
y =  \frac{t+1}{t-1}(1-x) + 1.
$$
One uses the relation
$$
2 - x^2 = \left(\frac{t+1}{t-1}(1-x) + 1\right)^2
$$
to determine the rational parametrization of $x^2 + y^2 = 2$:
$$
x=\frac{t^2+2t-1}{t^2+1},\quad y=\frac{t^2-2t-1}{t^2+1}.
$$
\end{proof}
\item (a) Let $a$, $b$, $c$, $d$, $e$, and $f$ be non-zero real numbers. Use the substitution $t = \tan(\theta/2)$ to transform the integral 
$$
\int \frac{a+b\cos\theta + c\sin\theta}{d+e\cos\theta + f\sin \theta}\,d\theta
$$
into the integral of a rational function in $t$.
\begin{proof}
If $t = \tan(\theta/2)$, then one finds that
$$
\cos\theta = \frac{1-t^2}{1+t^2} \quad\textrm{and}\quad \sin\theta = \frac{2t}{1+t^2}.
$$
Since
$$
d\theta = \frac{2dt}{1+t^2},
$$
one can determine that
$$
\int \frac{a+b\cos\theta + c\sin\theta}{d+e\cos\theta + f\sin \theta}\,d\theta = 2\int \frac{at^2-bt^2+2ct+a+b}{dt^4-et^4+2ft^3 + 2dt^2 + 2ft +d+e}\,dt.
$$
\end{proof}
(b) Evaluate the integral
$$
\int \frac{a+b\cos \theta + c\sin\theta}{1+\cos\theta+\sin\theta}\,d\theta.
$$
\begin{proof}
Using what was derived in (a), one attains an equivalent form of the integral in question:
$$
\int\frac{(a-b)t^2 + 2c t +a +b}{t^3+t^2+t+1}\,dt.
$$
This antiderivative comes out to
$$
(a-c)\log|t+1| + \frac{(c-b)\log(t^2+1)}{2} + (c+b)\arctan t + C.
$$
Substituting $t=\tan(\theta/2)$, one determines that the antiderivative in question evaluates to
$$
(a-c)\log\left|\tan(\theta/2) + 1\right| + \frac{(c-b)\log(\tan^2(\theta/2) +1) + (b+c)\theta}{2} + C.
$$
\end{proof}
\item For each of the following conics, either find a rational point or prove that there are no rational points.\\
(a)\qquad $x^2+y^2=6$
\begin{proof}
A theorem of Legendre states that an equation of the form
$$
aX^2 + bY^2 + cZ^2 = 0
$$
has a non-zero solution in the integers if and only if the following conditions hold:
\begin{enumerate}[label =\arabic*.]
\item$a$, $b$, and $c$ do not have the same sign.
\item $-bc$ is a square modulo $|a|$.
\item $-ac$ is a square modulo $|b|$.
\item $-ab$ is a square modulo $|c|$.
\end{enumerate}
Here, $a=1$, $b=1$, and $c=-6$. One notices that $-1$ is not a square modulo $6$, and so the conic in question has no rational points. One may also argue by Fermat's method of infinite descent \cite{rad6}.
\end{proof}
(b)\qquad $3x^2+5y^2=4$
\begin{proof}
One notices that since $20\equiv 2$ is not a quadratic residue modulo $3$, the conic in question has no rational points. 
\end{proof}
(c)\qquad $3x^2+6y^2=4$
\begin{proof}
One notices that since $18\equiv 2$ is not a quadratic residue modulo $4$, the conic in question has no rational points. 
\end{proof}
\item Prove that for every exponent $e\ge  1$, the congruence
$$
x^2+1\equiv 0\pmod{5^e}
$$
has a solution $x_e\in \Z/5^e\Z$. Prove further that these solutions can be chosen to satisfy
$$
x_1 \equiv 2\pmod 5 \quad \textrm{and}\quad x_{e+1} \equiv x_e \pmod {5^e} \textrm { for all } e \ge  1.
$$
(This is equivalent to showing that the equation $x^2+1=0$ has a solution in the $5$-adic numbers. It is a special case of Hensel's lemma. \textit{Hint.} Use induction on $e$.)
\begin{proof}
$x_1=2$, for example, is a solution in $\Z/5\Z$. One looks in $\Z/5^{e+1}\Z$ with the second constraint in mind. Suppose $m$ and $j$ are natural numbers. Then the following holds.
\begin{align*}
(5^em+x_e)^2 +1 	&\equiv 5^{2e}m^2+2x_e5^em+x^2_e + 1\\
				&\equiv 5^{e+1}5^{e-1}m^2 + 2x_e5^em+j5^e\\
				&\equiv (2x_em+j)5^e.
\end{align*}
$x_e$ is not a multiple of $5$, so one can choose an $m$ such that 
$$
2x_em\equiv -j\pmod {5}.
$$
The above expression is thus a multiple of $5^{e+1}$ as desired. For example,
$$
m=2jx_e^3
$$
suffices.
\end{proof}
\item Let $C_1$ and $C_2$ be the cubics given by the following equations:
$$
C_1 : x^3+2y^3-x-2y=0,\quad C_2:2x^3-y^3-2x+y=0.
$$
(a) Find the nine points of intersection of $C_1$ and $C_2$.
\begin{proof}
Setting $C_1=C_2$, one attains
$$
0 = x(x+1)(x-1) - 3y (y+1)(y-1).
$$
One sees that any of $x=0,\,\pm 1$ pairs with any of $y=0,\,\pm 1$ to yield a point of intersection. There are nine possibilities here, which exhausts all points of intersection. Therefore, the set of points of intersection is 
$$
\{(0,0),  (0,1),  (0,-1),  (1,0),  (1,1), (1,-1),  (-1,0), (-1,1), (-1,-1)\}.
$$
\end{proof}
(b) Let $\{ (0,0), P_1, \hdots, P_8 \}$ be the nine points from part (a). Prove that if a cubic curve goes through $P_1,\hdots,P_8$, then it must also go through the ninth point $(0,0)$.
\begin{proof}
The result follows from the Cayley-Bacharach theorem \cite{tao}. As Silverman and Tate put it, by B\'ezout's theorem, any two cubics with no common components defined in the projective plane with multiplicities of intersections and complex coordinates have $9$ points of intersection. They call ``the set of all cubics\ldots so to speak, nine dimensional.'' When one requires the cubic to go through a specified point, the set loses a dimension. So the family of cubics that intersect eight specified points is one dimensional. If $F(x,y)$ and $G(x,y)$ define two cubics, then one can find cubics going through the eight points by taking linear combinations $\lambda_1F(x,y) +\lambda_2G(x,y)$. It follows that the cubic in question has an equation 
$$
\lambda_1F(x,y) +\lambda_2G(x,y) = 0
$$
for suitable $\lambda_1$ and $\lambda_2$. Since $F(x,y)$ and $G(x,y)$ both vanish at the ninth point, so does $\lambda_1F(x,y) +\lambda_2G(x,y)$. Therefore, the cubic in question goes through all nine points, including $(0,0)$.
\end{proof}
%\item Define a composition law on the points of a cubic $C$ by the following rule as described in the text: Given $P,Q\in C$, then $P*Q$ is the point on $C$ so that $P$, $Q$, and $P*Q$ are collinear.\\
%(a) Explain why this law is commutative, $P*Q=Q*P$.
%\begin{proof}
%The line connecting $P$ with $Q$ is the same line conntecting $Q$ with $P$.
%\end{proof}
%(b) Prove that there is no identity element for this composition law; that is, there is no element $P_0\in C$ such that $P_0*P=P$ for all $P\in C$.
%\begin{proof}
%Suppose there is a $P_0\in C$ satisfying the property in question. Then the line connecting $P_0$ and $P$ is tangent to $P$ for all $P$, i.e. the tangent line to every point on $C$ also goes through $P_0$. This is absurd. So no $P_0$ satisfies this property.
%\end{proof}
%(c) Prove that this composition law is not associative; that is, in general $P*(Q*R) \neq (P*Q)*R$.
%\begin{proof}
%$P*(Q*R)$ is the point that is collinear to $P$ and also to the point that is collinear to $Q$ and $R$. $(P*Q)*R$ is the point that is collinear to the point that is collinear to $P$ and $Q$ and also to $R$. %finish
%\end{proof}
\setcounter{enumi}{10}
\item Let $S$ be a set with a composition law $*$ satisfying the following two properties:
\begin{enumerate}[label=(\roman*)]
\item \quad $P*Q=Q*P$ for all $P, Q\in S$.
\item \quad $P*(P*Q) = Q$ for all $P,Q\in S$. 
\end{enumerate}
Fix an element $\OO\in S$, and define a new composition law $+$ by the rule
$$
P+Q = \OO *(P*Q).
$$ 
(a) Prove that $+$ is commutative and has $\OO$ as identity element.
\begin{proof}
$+$ inherits commutativity from $*$:
\begin{align*}
Q + P 	&= \OO * (Q * P)\\
		&= \OO * (P * Q)\\
		&= P + Q.
\end{align*}
$\OO$ being the identity is corroborated below:
\begin{align*}
P+\OO 	&= \OO * (P * \OO)\\
		&= \OO * (\OO * P)\\
		&= P.
\end{align*}
\end{proof}
(b) Prove that for any given $P,Q\in S$, the equation $X+P=Q$ has the unique solution $X=P*(Q*\OO)$ in $S$. In particular, if we define $-P$ to be $P* (\OO*\OO)$, then $-P$ is the unique solution in $S$ of the equation $X+P=\OO$.
\begin{proof}
\begin{align*}
(P*(Q*\OO)) +  P	&= \OO * ((P*(Q*\OO)) * P)\\
				&= \OO * (P *(P*(Q*\OO)))\\
				&= \OO * (Q *\OO)\\
				&= \OO * (\OO * Q)\\
				&= Q.
\end{align*}
The unicity of this solution follows elementarily. % how?
\end{proof}
(c) Prove that $+$ is associative (and so $(S,+)$ is a group) if and only if 
$$
R * (\OO * (P *Q)) = P * (\OO *( Q * R))\quad \textrm {for all } P,Q,R\in S.
$$
\begin{proof}
\begin{align*}
P+(Q+R) &= (P+Q) + R\\
\OO * (P * (Q+R)) &= \OO * ((P+Q)* R)\\
\OO * (P*(\OO * (Q*R))) & = \OO * ((\OO * (P*Q))*R)\\
P*(\OO * (Q*R)) &= R * (\OO * (P*Q)).
\end{align*}
One sees that the condition of associativity is equivalent to the condition in question.
\end{proof}
%(d) Let $\OO'\in S$ be another point, and define a composition law $+'$ by $P+' Q = \OO' *(P*Q)$. Suppose that both $+$ and $+'$ are associative, so we obtain two group structures $(S, +)$ and $(S, +')$ on $S$. Prove that the map
%$$
%P \longmapsto \OO * (\OO' * P)
%$$
%is a (group) isomorphism from $(S, +)$ and $(S, +')$.
%\begin{proof}
%
%\end{proof}
%(e) Find a set $S$ with a composition law $*$ satisfying (i) and (ii) such that $(S,+)$ is not a group.
%\begin{proof}
%
%\end{proof}
\end{enumerate}

\section{Points of Finite Order}
\begin{enumerate}
\setlength{\parindent}{0pt}
\item Let $A$ be an abelian group, and for every integer $m\ge  1$, let $A_m$ be the set of elements $P\in A$ satisfying $mP=\OO$.\\
(a) Prove that $A_m$ is a subgroup of $A$.
\begin{proof}
Let $a,b\in A_m$, $|a|=r$, and $|b|=s$. Since 
$$
|a+b| \mid \textrm{lcm}(r,s),
$$
and $r\mid m$ and $s\mid m$ by assumption, 
$$
|a+b| \mid m.
$$ 
Since the order of an element's inverse is the same as the order of the element, and $\OO$ is always in $A_m$, $A_m$ is a subgroup of $A$.
\end{proof}
(b) Suppose that $A$ has order $M^2$, and further that for every integer $m$ dividing $M$, the subgroup $A_m$ has order $m^2$. Prove that $A$ is the direct product of two cyclic subgroups of order $M$. 
\begin{proof}
Suppose $a \neq \OO \in A_p$ for some prime $p\mid M$ where $|a| = p$ . Then 
$$
\langle a\rangle \subset A_p
$$
where $|\langle a\rangle| = p$. Suppose $b \neq \OO \in A_p$ and $b\notin \langle a \rangle$. Then 
$$
\langle b\rangle \subset A_p
$$
where $|\langle b\rangle| = p$. So $A_p = \langle a\rangle \oplus \langle b\rangle$, since by assumption $|A_p| =p^2$. This argument works because $A_p$ is a group of order $p^2$ in which for all $a\in A_p$, $pa = \OO$. Since $p$ is prime, $|a| = p$ for all $p\neq \OO$, so $a$ generates a cyclic group of order $p$. By the same argument and the condition that $|A_p| = p^2$, $A_p$ must be a direct product of $\langle a\rangle$ and another cyclic group of order $p$. 
%Suppose $m = p_1^{n_1} \cdots p_k^{n_k}$. Let $a_i$ be an element of order $p_i$. Then there is an element in $A_m$ whose order is the product of the $p_i$s (the exponent):
%\begin{align*}
%\left|  \sum_i a_i  \right| &= \textrm{lcm}(p_1,\hdots,p_k)\\
%				   &= \prod_i p_i.
%\end{align*}
\textit{One needs to show that for all $m$, there is an element of order $m$ in $A_m$, i.e. there is an element of order $M$ in $A$.}
\end{proof}
%(c) Find an example of a non-abelian group $G$ and an integer $m\ge  1$ so that the set $G_m = \{g\in G : g^m = e\}$ is not subgroup of $G$.
%\begin{proof}
%
%\end{proof}
\item Let $C$ be a non-singular cubic curve given by the usual Weierstrass equation
$$
y^2 = f(x) = x^3 +ax^2+bx+c.
$$
(a) Prove that
$$
\frac{d^2y}{dx^2} = \frac{2f''(x)f(x) - f'(x)^2}{4yf(x)} = \frac{\psi_3(x)}{4yf(x)}.
$$
Use this to deduce that a point $P = (x,y)\in C$ is a point of order three if and only if $P\neq\OO$ and $P$ is a point of inflection on the curve $C$.
\begin{proof}
One finds that
$$
\frac{d^2y}{dx^2} = \frac{2y^2(6x+2a) -(3x^2+2ax+b)^2}{4y^3}.
$$
According to Silverman and Tate (p. 40), the roots of $\psi_3(x)$ correspond to the points of order three on $C$. The points of order three, therefore, are the points of inflection.
\end{proof}
(b) Suppose now that $a$, $b$, and $c$ are in $\R$. Prove that $\psi_3(x)$ has exactly two real roots, say $\alpha_1$ and $\alpha_2$ with $\alpha_1<\alpha_2$. Prove that $f(\alpha_1)<0$ and $f(\alpha_2)>0$. Use this to deduce that the points in $C(\R)$ of order dividing $3$ form a cyclic group of order three.
\begin{proof}
One should note that $\psi_3(x)$ has four distinct complex roots. A quartic polynomial with real coefficients and a positive leading coefficient has exactly two real roots if it takes negative values at all roots of its derivative. Silverman and Tate note that
$$
\psi_3' (x) = 12f(x),
$$
so the roots of $\psi_3' (x)$ correspond to the roots of $f(x)$. Using the fact that 
$$
\psi_3(x) = 2f''(x)f(x) - f'(x)^2,
$$
and setting $f(x)=0$, one finds that 
$$
\psi_3(x) < 0
$$
when $\psi_3'(x)=0$. So $\psi_3(x)$ has exactly two real roots. Silverman and Tate note that if $m$ is odd, the points of order dividing $m$ in $C(\R)$ form a cyclic group of order $m$. So in the case $m=3$, one attains a cyclic group of order $3$, with the point at infinity being the identity. The other two points correspond to $(\alpha_2, \pm \delta_2)$, where $\alpha_2$ is the real root of $\psi_3(x)$ greater than zero at $f$ and $\delta_2 =\sqrt{f(\alpha_2)}$. There are no other points of order dividing $3$ in $C(\R)$, which implies that $\alpha_1$, though it is real, satisfies the following:
$$
y^2 = f(\alpha_1) < 0.
$$
If this wasn't the case, then $\alpha_1$ would correspond to two other points of order three. So the second requirement holds. 
% does $\sqrt{\textrm{cubic}} \implies$ $1$ point of inflection?
\end{proof}
\item Let $\omega_1,\omega_2\in\C$ be two complex numbers which are $\R$--linearly independent, and let 
$$
L = \Z\omega_1 + \Z\omega_2 = \{n_1\omega_1 + n_2\omega_2 : n_1,n_2\in\Z\}
$$
be the lattice of $\C$ that they generate.\\
(a) Show that the series
$$
\wp (u) = \frac{1}{u^2} + \sum_{\substack{\omega\in L\\\omega\neq 0}}\left(  \frac{1}{(u-\omega)^2} - \frac{1}{\omega^2} \right)
$$
defining the Weierstrass $\wp$ function is absolutely and uniformly convergent on any compact subset of the complex $u$ plane which does not contain any of the points of $L$. Conclude that $\wp$ is a meromorphic function with a double pole at each point of $L$ and that $\wp$ has no other poles. \textit{Please come back to this question after more experience in analysis.}
% uniform convergence: if $E$ is a set and $(f_n)_{n\in \N}$ is a sequence of real-valued functions on $E$, $(f_n)_{n\in \N}$ is uniformly convergent on $E$ with limit $f : E\to \R$ if for every $\epsilon > 0$ there exists $$  |f_n(x)-f(x)| < \epsilon.  $$
% uniform absolute convergence: the series $\sum_n f_n(x)$ is uniformly convergent if $\sum_n |f_n(x)|$ converges absolutely.  
% a compact subset of the complex plane is one which is closed and bounded.
% a holomorphic function is one that is at every point of its sub domain complex differentiable in a neighbourhood of the point
% a meromorphic function is holomorphic everywhere except for its poles

(b) Prove that $\wp$ is an even function.
\begin{proof}
Putting in $-u$ for $u$,
\begin{align*}
\wp (-u) &= \frac{1}{(-u)^2} + \sum_{\substack{\omega\in L\\\omega\neq 0}}\left(  \frac{1}{(-u-\omega)^2} - \frac{1}{\omega^2} \right)\\
&=\frac{1}{u^2} + \sum_{\substack{\omega\in L\\\omega\neq 0}}\left(  \frac{1}{(u+\omega)^2} - \frac{1}{\omega^2} \right).
\end{align*}
One notices that if $\omega\in L$, then so is $-\omega$. So it is valid to replace $\omega$ with $-\omega$:
\begin{align*}
\wp (-u) &= \frac{1}{u^2} + \sum_{\substack{\omega\in L\\\omega\neq 0}}\left(  \frac{1}{(u-\omega)^2} - \frac{1}{\omega^2} \right)\\
&=\wp(u).
\end{align*}
So $\wp$ is even.
\end{proof}
(c) Show that $\wp$ is a doubly periodic function; that is, show that
$$
\wp(u+\omega) = \wp(u) \textrm{ for every } u\in \C \textrm{ and every } \omega \in L.
$$
(\textit{Hint.} From (a), you can calculate the derivative $\wp'(u)$ by differentiating each term of the series defining $\wp(u)$. First prove $\wp ' (u+\omega) = \wp'(u)$, then integrate.)
\begin{proof}
One finds that
$$
\wp'(u) = -2\sum_{\omega\in L} \frac{1}{(u-\omega)^3}.
$$
Putting in $u+\omega_1$ for $u$ where $\omega_1\in L$,
\begin{align*}
\wp'(u+\omega_1) &= -2\sum_{\omega\in L} \frac{1}{(u+\omega_1-\omega)^3}.
\end{align*}
One notices that since both $\omega$ and $\omega_1$ are in the lattice, so is their difference. Furthermore, since $\omega_1$ is constant and $\omega$ is variable over all of $L$, their difference is the entire lattice. Therefore,
\begin{align*}
\wp'(u+\omega_1) &= -2\sum_{\omega\in L} \frac{1}{(u+\omega)^3}\\
				&=  -2\sum_{\omega\in L} \frac{1}{(u-\omega)^3}\\
				&= \wp'(u).
\end{align*}
Thus,
$$
\wp'(u+\omega_1) - \wp'(u) = 0
$$
so for some constant $a$
$$
\wp(u+\omega_1) - \wp(u) = a.
$$
Setting $u = -\omega_1/2$,
\begin{align*}
\wp(-\omega_1/2+\omega_1) - \wp(-\omega_1/2) &= a\\
\wp(\omega_1/2) - \wp(-\omega_1/2) &= a\\
\wp(\omega_1/2) - \wp(\omega_1/2) &= a\\
0&=a.
\end{align*}
So $\wp$ is doubly periodic.
\end{proof}
\item Let $C$ be the cubic curve 
$$
y^2 = x^3+1.
$$
(a) For each prime $5\le  p < 30$, describe the group of points on this curve having coordinates in the finite field with $p$ elements.
\begin{proof}
A simple Python script with output below outputs the points on $C$ over $\F_p$ for a given $p$ as well as $N_p$: the total number of points listed adjoin the point at infinity.
\footnotesize
\begin{spverbatim}
5
[(0, 1), (0, 4), (2, 2), (2, 3), (4, 0)] 
N_p = 6
7
[(0, 1), (0, 6), (1, 3), (1, 4), (2, 3), (2, 4), (3, 0), (4, 3), (4, 4), (5, 0), (6, 0)] 
N_p = 12
11
[(0, 1), (0, 10), (2, 3), (2, 8), (5, 4), (5, 7), (7, 5), (7, 6), (9, 2), (9, 9), (10, 0)] 
N_p = 12
13
[(0, 1), (0, 12), (2, 3), (2, 10), (4, 0), (5, 3), (5, 10), (6, 3), (6, 10), (10, 0), (12, 0)] 
N_p = 12
17
[(0, 1), (0, 16), (1, 6), (1, 11), (2, 3), (2, 14), (6, 8), (6, 9), (7, 2), (7, 15), (9, 4), (9, 13), (10, 7), (10, 10), (14, 5), (14, 12), (16, 0)] 
N_p = 18
19
[(0, 1), (0, 18), (2, 3), (2, 16), (3, 3), (3, 16), (8, 0), (12, 0), (14, 3), (14, 16), (18, 0)] 
N_p = 12
23
[(0, 1), (0, 22), (1, 5), (1, 18), (2, 3), (2, 20), (10, 9), (10, 14), (12, 2), (12, 21), (13, 6), (13, 17), (14, 10), (14, 13), (15, 8), (15, 15), (16, 7), (16, 16), (19, 11), (19, 12), (21, 4), (21, 19), (22, 0)] 
N_p = 24
29
[(0, 1), (0, 28), (2, 3), (2, 26), (3, 12), (3, 17), (4, 6), (4, 23), (7, 5), (7, 24), (8, 7), (8, 22), (9, 11), (9, 18), (13, 9), (13, 20), (17, 10), (17, 19), (18, 2), (18, 27), (19, 4), (19, 25), (22, 8), (22, 21), (25, 13), (25, 16), (27, 14), (27, 15), (28, 0)] 
N_p = 30
\end{spverbatim}
\normalsize
\end{proof}
(b) For each prime in (a), let $M_p$ be the number of points in the group. (Don't forget the point at infinity.) For the set of primes satisfying $p\equiv 2\pmod 3$, can you see a pattern for the values of $M_p$? Make a general conjecture for the value of $M_p$ when $p\equiv 2\pmod 3$ and prove that your conjecture is correct.
\begin{proof}
The primes $p$ between $5$ and $30$ with $p\equiv 2\pmod 3$ are $5$, $11$, $17$, $23$, and $29$. One notices that for each $p$ with this property, 
$$M_p = p+1.$$
One tries to search for counterexamples, but none seem to show up. One notes that $3$ does not divide the order of $\F_p^*$ in the case in question. Therefore, $x\mapsto x^3$ is an $\F_p^*$-automorphism. Therefore, the equation
$$
y^2 = x + 1
$$
is equally valid over $\F_p$. One chooses an element in $\F_p$ for $y$; the fact that only one corresponding $x$ exists is apparent. Thus, there are $p$ solutions for elements in $\F_p$, and, with the point at infinity, 
$$
M_p = p + 1
$$
for all $p\equiv 2\pmod 3$.
\end{proof}
\item Let $f(x)= x^2+ax+b = (x-\alpha_1)(x-\alpha_2)$ be a quadratic polynomial with the indicated factorization. Prove that
$$
(\alpha_1-\alpha_2)^2 = a^2 - 4b.
$$
\begin{proof}
One finds that $b = \alpha_1\alpha_2$ and $a = -(\alpha_1 + \alpha_2)$; an easy way to reach such a conclusion is to think of how one factors $f(x)$ in terms of $a$ and $b$. It follows that
\begin{align*}
a^2 -4b	&= (-(\alpha_1+\alpha_2))^2 - 4\alpha_1\alpha_2\\
		&= (\alpha_1 + \alpha_2)^2 - 4\alpha_1\alpha_2\\
		&= \alpha_1^2 + 2\alpha_1\alpha_2 + \alpha_2^2 - 4\alpha_1\alpha_2\\
		&= \alpha_1^2 - 2\alpha_1\alpha_2 + \alpha_2^2\\
		&= (\alpha_1 - \alpha_2)^2.
\end{align*}
So the proof is complete.
\end{proof}
(b) Let $f(x) = x^3+ax^2+bx+c=(x-\alpha_1)(x-\alpha_2)(x-\alpha_3)$ be a cubic polynomial with the indicated factorization. Prove that
$$
(\alpha_1-\alpha_2)(\alpha_1-\alpha_3)(\alpha_2-\alpha_3) = -4a^3c +a^2b^2 + 18abc - 4b^3 - 27c^2.
$$
\begin{proof}
One finds that $a = -(\alpha_1+\alpha_2+\alpha_3)$, $b = \alpha_1\alpha_2 + \alpha_1\alpha_3 + \alpha_2\alpha_3$, and $c=-\alpha_1\alpha_2\alpha_3$.  Using these relations, one can prove (as is done in \texttt{cubic\_discriminant.pdf}) the equality in question.
\end{proof}
(c) Let
$$
f(x) = x^n + a_1x^{n-1} +\cdots + a_n = (x-\alpha_1)(x-\alpha_2)\cdots(x-\alpha_n)
$$
be a polynomial with the indicated factorization. The \textit{discriminant of $f$} is defined to be
$$
\textrm{Disc}(f) =\prod_{i=1}^{n-1}\prod_{j=i+1}^n (\alpha_i-\alpha_j)^2,
$$
so $\textrm{Disc}(f)=0$ if and only if $f$ has a double root. Prove that $\textrm{Disc}(f)$ can be expressed as a polynomial in the coefficients $a_1,\hdots , a_n$ of $f$. 
\begin{proof}
By Vieta's formulas \cite{vieta}, one obtains a system of $n$ distinct equations in the $n$ roots of $f(x)$ in terms of the coefficients of $f(x)$. Therefore, one can solve for each root of the polynomial in terms of the coefficients, substitute these values into the formula for the discriminant, and obtain the desired polynomial.
\end{proof}
\item Let $p$ be a prime, and for a rational number $r = (m/n)p^\nu$ with $m,n$ prime to $p$, let $\ord(r) = \nu$ be as in the text. [By convention, we will set $\ord(0)=\infty$.]\\
(a) Prove that for all rational numbers $r_1$ and $r_2$,
$$
\ord (r_1r_2) = \ord(r_1) + \ord(r_2).
$$
\begin{proof}
If $r_1=0$ (without loss of generality), then the identity in question holds, for
\begin{align*}
\ord(0\cdot r_2) &= \ord(0) + \ord(r_2)\\
\ord(0)		&=\ord(0) + \ord(r_2)\\
\infty			&= \infty + \ord(r_2)\\
\infty			&= \infty.
\end{align*}
If $r_1=(m_1/n_1)p^{k_1}$ and $r_2 = (m_2/n_2)p^{k_2}$, then the identity also holds:
\begin{align*}
\ord\left(\frac{m_1}{n_1}p^{k_1} \cdot \frac{m_2}{n_2}p^{k_2} \right) 	&= \ord\left(\frac{m_1}{n_1}p^{k_1}\right) + \ord\left(\frac{m_2}{n_2}p^{k_2}\right)\\
\ord\left(  \frac{m_1m_2}{n_1n_2}p^{k_1+k_2}  \right)				&= k_1 + k_2\\
k_1+k_2												&= k_1 + k_2.
\end{align*}
\end{proof}
(b) Prove that for all rational numbers $r_1$ and $r_2$,
$$
\ord(r_1+r_2) \ge  \min\{  \ord(r_1),\ord(r_2)  \}.
$$
Further, if $\ord(r_1) \neq\ord(r_2)$, prove that the inequality is an equality.
\begin{proof}
If $\ord(r_1) = \ord(r_2)$, one gets the following:
\begin{align*}
\ord\left(\frac{m_1}{n_1}p^{k} + \frac{m_2}{n_2}p^{k} \right) &\ge  \min\left\{    \ord\left(\frac{m_1}{n_1}p^{k}\right) , \ord\left(\frac{m_2}{n_2}p^{k}\right)\right\}\\
\ord\left(p^k\left(\frac{m_1}{n_1} + \frac{m_2}{n_2} \right)\right) & \ge  \min\{k,k\}\\
\ord\left(p^k\left(\frac{m_1n_2 + m_2n_1}{n_1n_2} \right)\right) &\ge  k\\
k+\ord\left(\frac{m_1n_2 + m_2n_1}{n_1n_2} \right) &\ge  k.
\end{align*}
$n_1n_2$ must be prime to $p$, but $m_1n_2 + m_2n_1$ does not need to be prime to $p$. So
$$
\ord\left(\frac{m_1n_2 + m_2n_1}{n_1n_2} \right) = \ell \ge  0,
$$
and thus
$$
k + \ell \ge  k
$$
holds, and so does the original inequality. If $\ord(r_1) \neq\ord(r_2)$, and (without loss of generality) $\ord(r_1) >\ord(r_2)$, one attains the following:
\begin{align*}
\ord\left(\frac{m_1}{n_1}p^{k_1} + \frac{m_2}{n_2}p^{k_2} \right) &\ge  \ord(r_2)\\
\ord\left(p^{k_1} + \frac{m_2n_1}{m_1n_2}p^{k_2} \right) &\ge  k_2\\
\ord\left(p^{k_1} + \frac{m_3}{n_3}p^{k_2} \right) &\ge  k_2,
\end{align*}
where $m_3$ and $n_3$ are prime to $p$. Then, using the fact that $k_1 - k_2 >0$,
\begin{align*}
\ord\left(p^{k_2}\left( p^{k_1-k_2} + \frac{m_3}{n_3}\right) \right) &\ge  k_2\\
k_2 + \ord\left( p^{k_1-k_2} + \frac{m_3}{n_3}\right) &\ge  k_2\\
\ord\left( \frac{n_3p^{k_1-k_2} + m_3}{n_3}\right) &\ge  0.
\end{align*}
Since $p$ divides $p^{k_1-k_2}$ but does not divide $m_3$, $p$ cannot divide the numerator. Similarly, $p$ does not divide $n_3$, so it cannot divide the denominator. Therefore,  
$$
\ord\left( \frac{n_3p^{k_1-k_2} + m_3}{n_3}\right) = 0
$$
and the inequality is an equality as desired.
\end{proof}
(c) Define an ``absolute value'' on the rational numbers by the rule 
$$
\|r\| = \frac{1}{p^{\ord(r)}}.
$$
(By convention, we set $\|0\|=0.)$ Prove that $\|\cdot\|$ has the following properties:\\
\begin{enumerate}[label = (\roman*)]
\item $\| r\| \ge  0$; and $\| r\| =0$ if and only if $r=0$.
\item $\|r_1r_2\| = \|r_1\| \cdot \|r_2\|$.
\item $\|r_1+r_2\| \le  \max\{   \|r_1\|,\|r_2\|  \}$.
\end{enumerate}
\textrm{}\\
Notice that property (iii) is stronger than the usual triangle inequality. The absolute value $\| \cdot\|$ is called the \textit{$p$-adic absolute value} on the rational numbers. It can be used to define a topology on the rational numbers, the \textit{$p$-adic topology}.
\begin{proof}
In each proof, the $m_i$ and $n_i$ are prime to $p$. (i) $\ord(r)$ is an integer for all $r\neq 0$. $1/p^n > 0$ for all integers $n$. By convention, $\| 0 \| =0$, so both conditions are met.\\
(ii) If $r_1=(m_1/n_1)p^{k_1}$ and $r_2 = (m_2/n_2)p^{k_2}$, then the identity in question holds:
\begin{align*}
\left\| \frac{m_1}{n_1}p^{k_1} \cdot \frac{m_2}{n_2}p^{k_2} \right\| & = \left\| \frac{m_1}{n_1}p^{k_1} \right\| \cdot \left\| \frac{m_2}{n_2}p^{k_2} \right\|\\
\left\| \frac{m_1m_2}{n_1n_2}p^{k_1+k_2} \right\| & =\frac{1}{p^{k_1}} \cdot \frac{1}{p^{k_2}}\\
\frac{1}{p^{k_1+k_2}} &= \frac{1}{p^{k_1+k_2}} .
\end{align*}
(iii) Suppose $\|r_1\| = \|r_2\|$. Then the inequality in question holds:
\begin{align*}
\left\|\frac{m_1}{n_1}p^{k}+\frac{m_2}{n_2}p^{k}\right\|& \le  \max\left\{   \left\|\frac{m_1}{n_1}p^{k}\right\| , \left\|\frac{m_2}{n_2}p^{k}\right\| \right\}\\
\left\|p^k \left(\frac{m_1}{n_1}+\frac{m_2}{n_2}\right)\right\|& \le  \max\left\{   \frac{1}{p^{k}} , \frac{1}{p^k}\right\}\\
\frac{1}{p^k} &\le  \frac{1}{p^k}.
\end{align*}
Suppose (without loss of generality) that $\|r_1\|>\|r_2\|$. Then, using the fact that $k_1 < k_2$, one obtains
\begin{align*}
\left\|\frac{m_1}{n_1}p^{k_1}+\frac{m_2}{n_2}p^{k_2}\right\|& \le  \max\left\{   \left\| r_1\right\| ,\left \|r_2\right\|\right\}\\
\left\|\frac{m_1}{n_1}p^{k_1} \left(1+\frac{m_2n_1}{m_1n_2}p^{k_2-k_1}\right)\right\|& \le  \|r_1\|\\
\left\|\frac{m_1}{n_1}p^{k_1}\right\| \cdot \left\|\frac{m_1n_2 + m_2n_1p^{k_2-k_1}}{m_1n_2}\right\|& \le  \|r_1\|
\end{align*}
Neither the numerator nor the denominator of the second term of the left hand side is divisible by $p$, for $p$ divides $p^{k_2-k_1}$ and is prime to the $m_i$ and $n_i$ by assumption. Therefore,
\begin{align*}
\left\|r_1\right\| \cdot \frac{1}{p^0}& \le  \|r_1\|\\
\left\|r_1\right\| &\le  \| r_1\|.
\end{align*}
The inequality holds. % Am I NUTS?
\end{proof}
\item Let $p$ be a prime, and let $R=R_p$ be the set described in the text, 
\begin{align*}
R &= \{\textrm{non-zero rational numbers $x$ such that } \ord(x)\ge  0\}\cup \{0\}\\
&= \{x\in\Q : \|x\| \le  1\}.
\end{align*}
(Here $\|\cdot\|$ is the $p$-adic absolute value defined in the previous exercise. So the set $R$ is a $p$-adic analogue of the interval $[-1,1]$ on the real line or of the unit disk $\{z\in\C : |z| \le  1\}$ in the complex plane.)\\
(a) Prove that $R$ is a subring of the rational numbers.
\begin{proof}
$1\in R$ and $0\in R$. For all $r\in R$, $-r \in R$ since $\ord(r) = \ord(-r)$. The sum of two elements $r_1, r_2\in R$ is in $R$ since it was proved that
$$
\ord(r_1+r_2) = \min\{\ord(r_1), \ord(r_2)\}.
$$ 
The product of two elements in $R$ is also in $R$ for it was proved that
$$
\ord(r_1r_2) = \ord(r_1) + \ord(r_2).
$$
So $R$ is a subring of $\Q$. % Missing anything?
\end{proof}
(b) Prove that the ideal generated by $p$ is a maximal ideal, and describe the quotient field $R/pR$.
\begin{proof}
Suppose 
$$
(p) \subset I \subset R.
$$
Since for all $r \in (p)$ one has $\ord(r) \ge  1$, there must be some $\ell\in I$ for which $\ord(\ell) = 0$. Therefore, $\ord(\ell^{-1}) = 0$, so $\ell \ell^{-1}\in I$. So $1\in I$, and therefore $I=R$. So $(p)$ is maximal. (Another approach.) If $R/(p)$ is a field, then $(p)$ is maximal. One notices that
$$
R/(p) = \{ r\in R : \ord(r)=0\} \cup \{0\}.
$$
This quotient is clearly closed under multiplication and closed under addition as one takes the numerator of the sum of two numbers mod $p$. It has additive inverses (if $a = m/n$, take $-a = (-m)/n$) and $(R/(p))^*$ has multiplicative inverses (take $a^{-1} = n/m$). $R/(a)$ is not a field for an arbitrary $a\in R$, since $\ord(a^{-1}) < 0$ in some cases. It only works since one mods out by $p$. Therefore, $R/(p)$ is a field and $(p)$ is maximal. % describe?
\end{proof}
(c) Prove that the unit group of $R$ consists of all rational numbers $a/b$ such that $p$ does not divide $ab$.
\begin{proof}
A unit is an element with a multiplicative inverse. One notes that
$$
R = \{  m/n\in \Q : p \nmid n \}.
$$
Therefore, for $r = m/n\in R$ to have a multiplicative inverse, $p$ cannot divide $m$ either, for if $p$ divides $m$, $r^{-1} = n/m$, so $p$ divides the denominator of $r^{-1}$. In that case, $r^{-1}\notin R$. So the units of $R$ are precisely the rational numbers $a/b$ where $p$ does not divide $ab$.
\end{proof}
(d) Prove that $R$ is a unique factorization domain.
\begin{proof}
Recall that
$$
R^* = \{ r\in R : \ord(r) = 0\}.
$$
By the definition of $R$, any $r\in R$ can be written as the product of a unit and a power of $p$. $p$ is irreducible, so the factorization of anything in $R$ into the product of a unit and a power of $p$ is unique. So $R$ is a UFD.
\end{proof}
%(e) Describe all of the ideals of $R$. Use this description to prove that the ideal generated by $p$ is the only maximal ideal of $R$. (A ring such as $R$, which has exactly one maximal ideal. is called a \textit{local ring}.) %no idea
%\item Let $p$ and $R$ be as in the previous exercise, and let $\sigma \ge  \nu$ be integers. Prove that the quotient group $p^\nu R/p^\sigma R$ is a cyclic group of order $p^{\sigma - \nu}$.
%\begin{proof}
%% Intuitive, but how to prove?
%\end{proof}
%\item Let $p$ be a prime, and let $S=S_p$ be the set of rational numbers whose denominator is a power of $p$. ($p^0=1$ is allowed.) Thus, $S$ consists of all rational numbers $ap^\nu$, where $a$ is an integer prime to $p$ and $\nu$ is an arbitrary integer.\\
%(a) Prove that $S$ is a subring of the rational numbers.
%\begin{proof}
%
%\end{proof}
%(b) Prove that the unit group in $S$ consists of all numbers $\pm p^\nu$ with $\nu$ any integer.
%\begin{proof}
%
%\end{proof}
%(c) Let $q$ be a prime other than $p$. Prove that $q$ generates a maximal ideal of $S$, and describe the quotient field $S/qS$. Prove that every maximal ideal of $S$ is of this form.
%\begin{proof}
%
%\end{proof}
\setcounter{enumi}{9}
\item Let $p\ge  2$ be a prime and let $C$ be the cubic curve 
$$
C : y^2 = x^3 + px.
$$
Find all points of finite order in $C(\Q)$.
\begin{proof}
By Nagell--Lutz, one finds that the possible values for $y$ are $0$, $\pm1$, $\pm2$, $\pm p$, and $\pm 2p$ (as their squares divide $D = -4p^3$). Setting $y=0$, one determines that $(0,0)$ is the only solution; this solution corresponds to the only point of order $2$. Setting $y=1$, under no circumstance does 
$$
1 = x^3 + px
$$ 
have solutions in the integers. Setting $y=2$, one finds a solution to
$$
4= x^3+px
$$
when $p=3$, namely $(1,2)$. However, $2(1,2) $ does not have integer coordinates, so $(1,2)$ is not of finite order. This equation has no other solutions in the integers for all primes $p$ since $x^3 + px$ exceeds $4$ for all primes $p>3$ with the condition that $x\in \Z$. Setting $y=p$, one gets the equation
$$
p^2 = x^3 + px.
$$
Suppose $p$ is odd. Reducing mod $2$, one obtains
\begin{align*}
0 &= x + x + 1\\
	&= 1.
\end{align*}
So there are no solutions for odd $p$. Suppose $p=2$. Then 
$$
4 = x^3 + 2x.
$$
This equation has no integer solutions. Lastly, set $y = 2p$. One has
$$
4p^2 = x^3 + px.
$$
It follows that $p$ divides $x$. Since $x>0$, $x \ge  p$. One gets 
$$
x^3 + px \ge  p^3 + p^2.
$$
Therefore, $4p^2 \ge  p^3 + p^2$, so $p \le  3$. Setting $p=2$, one gets
$$
8 = x^3 +  2x,
$$
which has no integer solutions. Setting $p=3$, one gets
$$
36 = x^3 + 3x.
$$
One finds that $x= 3$ is the only integer solution, so $(3,6)$ is a potential point of finite order. However, $2(3,6)$ does not have integer coordinates, so $(3,6)$ is of finite order. Having exhausted all possibilities, one finds that the only rational points of finite order on $C$ are $(0,0)$ and $\OO$.
\end{proof}
\setcounter{enumi}{11}
\item For each of the following cubic curves, determine all of the points of finite order. Also determine the structure of the group formed by these points.\\
(a) $y^2 = x^3 - 2$.
\begin{proof}
Let $G$ be the subgroup of $C(\Q)$ consisting of all points of finite order. Silverman and Tate prove that
$$
G \longrightarrow \tilde C(\F_p) \quad:\quad P \longmapsto \tilde P = \begin{cases} (\tilde x, \tilde y) & P = (x,y)\\ \tilde \OO & P = \OO\end{cases}
$$
is an isomorphism onto a subgroup of $\tilde C(\F_p)$ if $p$ doesn't divide $2D$, where $D$ is the discriminant of $C$. In this case, one finds that $D = -108 = (-1) \times2^2 \times 3^3$. Choosing $2$ primes greater than $3$, one has
$$
\# \tilde C(\F_5) = 6\quad\textrm{and}\quad\# \tilde C(\F_7) = 7.
$$
So $\#G$ divides $6$ and $7$, and thus $\#G = 1$. So the only point of finite order is $\OO$ and $G$ is the trivial group.
\end{proof}
(b) $y^2 = x^3 + 8$.
\begin{proof}
Here, the discriminant $D = -1728 = (-1)\times 2^6\times 3^3$. One finds that
$$
\# \tilde C(\F_5) = 6\quad\textrm{and}\quad\# \tilde C(\F_{13}) = 16.
$$
So $\# G$ is at most $2$. However, one notices that $(-2, 0)$ is a solution of $C$ of order $2$. Therefore,
$$
G = \{ \OO, (-2,0)\} \isomto \Z/2\Z
$$
is cyclic of order $2$.
\end{proof}
(c) $y^2 = x^3 + 4$.
\begin{proof}
The discriminant of $C$ is $D = -432 = (-1)\times 2^4\times 3^3$. One has
$$
\# \tilde C(\F_7) = 3
$$ 
so $\# G$ is either $1$ or $3$. One notices that $(0, \pm 2)$ are points on the curve, and the horizontal coordinate of $2(0,2)$ is $0$ and that of $2(0,-2)$ is also $0$; therefore, they are points of order $3$ and comprise the group, along with the point at infinity:
$$
G = \{\OO, (0,2), (0,-2)\} \isomto \Z/3\Z.
$$
\end{proof}
(d) $y^2 = x^3 + 4x$.
\begin{proof}
One has $D = -256 = (-1)\times 2^8$, and one finds that 
$$
\# \tilde C(\F_3) = 4.
$$
So $\# G$ is $1$, $2$, or $4$. $(0,0)$ is a manifest solution of order $2$. $(2,\pm 4)$ are points on the curve, and one finds that $2(2,4) = (0,0)$, $4(2,4) = \OO$, $2(2,-4) = (0,0)$, and $4(2,-4) = \OO$. So $G$ is cyclic of order $4$:
$$
G = \{\OO, (0,0), (2,4), (2,-4)\} \isomto \Z/4\Z.
$$
\end{proof}
(f) $y^2 = x^3 + 1$.
\begin{proof}
$D = -27 = (-1)\times 3^3$. One notices (inconsequentially) that for the first few values of $p$, $\#\tilde C (\F_p)$ is divisible by $6$. Using Nagell--Lutz, the $y$-coordinates of the finitely-ordered points must be $0$, $\pm 1$, or $\pm 3$. One has a point of order $2$ in $(-1, 0)$. $(0, \pm 1)$ are points on the curve, and one finds that horizontal coordinate of $2(0, 1)$ is $0$, and that of $2(0, -1)$ is also $0$. Therefore, $(0,\pm 1)$ are points of order $3$. $(2,\pm 3)$ are also on the curve. One finds that $2(2,3)=(0,1)$, so $(2,\pm 3)$ (by symmetry) are of order $6$. Thus it has been shown that
$$
G = \{\OO, (-1,0), (0,\pm1), (2,\pm 3)\} \isomto \Z/6\Z.
$$
\end{proof}
(g) $y^2 = x^3 -43x + 166$.
\begin{proof}
$D = -425984 = (-1)\times 2^{15} \times 13$. One finds that
$$
\# \tilde C(\F_3) = 7,
$$
so $\# G$ is $1$ or $7$. One finds that $(3,\pm 8)$ is on $C$. Taking multiples of $(3,8)$,
\begin{align*}
2(3,8) &= (-5, -16)\\
3(3,8) &= (11, -32)\\
4(3,8) & = (11, 32)\\
5(3,8) &= (-5,16)\\
6(3,8) &= (3, -8)\\
7(3,8) &= \OO.
\end{align*}
Taking all multiples of $(3,8)$ was not necessary, but it holds that
$$
G=\{ \OO, (3,\pm 8), (-5,\pm 16), (11,\pm 32)\} \isomto \Z/7\Z.
$$
\end{proof}
(l) $y^2 = x^3 - 4x$.
\begin{proof}
$D =  256  = 2^8$, and one finds that 
$$
\# \tilde C(\F_3) = 4.
$$
So $\# G$ is $1$, $2$, or $4$. $(0,0)$ is a solution of order $2$. One also finds that $(\pm2, 0)$ are solutions of order $2$. Since the group is of order $4$ and there are no points of order $4$, $G$ must be isomorphic to the $4$-group:
$$
G = \{ \OO, (0,0), (\pm 2, 0)\}\isomto \Z/2\Z \times \Z/2\Z.
$$
\end{proof}
\end{enumerate}
\section{The Group of Rational Points}
\begin{enumerate}
\setlength{\parindent}{0pt}
%\item (a) Prove that the set of rational numbers $x$ with height $H(x)$ less than $\kappa$ contains at most $2\kappa^2 + \kappa$ elements.  
%\begin{proof}
%
%\end{proof}
%(b) Let $R(\kappa)$ be the set of rational numbers $x$ with height $H(x)$ less than $\kappa$. Prove that 
%$$
%\lim_{n\to\infty} \frac{\# R(\kappa)}{\kappa^2} = \frac{12}{\pi^2}. % probability of numbers being coprime?
%$$
%\begin{proof}
%
%\end{proof}
%\setcounter{enumi}{2}
%\item Let $C$ be a rational cubic curve given by the usual Weierstrass equation. \\
%(a) Prove that for any rational point $P\in C(\Q)$, the limit
%$$
%\hat h(P) = \lim_{n\to\infty} \frac{1}{4^n}h(2^n P)
%$$
%exists. The quantity $\hat h(P)$ is called the \textit{canonical height of $P$}. (\textit{Hint.} Try to prove that the sequence is Cauchy.)\begin{proof}
%$$
%\exists\kappa : \hat h(P) \ge  \lim_{n\to\infty} h(P) - \frac{\kappa}{4^n} \forall P \in C(\Q).
%$$
%\end{proof}
%(b) Prove that there is a constant $\kappa$, depending only on $a$, $b$, $c$, so that for all rational points $P$ we have
%$$
%-\kappa \le  \hat h(P)-h(P) \le  \kappa.
%$$ 
%\begin{proof}
%
%\end{proof}
%(c) Prove that for any integer $m$ and any rational point $P$, 
%$$
%\hat h(mP) = m^2\hat h(P).
%$$
%\begin{proof}
%
%\end{proof}
%(d) Prove that $\hat h(P) = 0$ if and only if $P$ is a point of finite order.
%\begin{proof}
%
%\end{proof}
\setcounter{enumi}{6}
\item This exercise describes a variant of the Nagell--Lutz theorem which often simplifies calculations on curves with a rational point of order two.\\
%(a) Let $C$ be a non-singular cubic curve given in Weierstrass form by an equation
%$$
%C :y^2=x^3+ax^2+bx
%$$
%where $a$ and $b$ are integers. Let $P=(x,y)\in C(\Q)$ be a point of finite order with $y\neq 0$. Prove that $x$ divides $b$ and that the quantity
%$$
%x + a + \frac{b}{x}
%$$
%is a perfect square. [Note that if this quantity is a square, say equal to $N^2$, then $(x,xN)$ is a rational point on $C$; but such a point need not have finite order. So this exercise gives a necessary condition for $P$ to have finite order, but not a sufficient condition.]
%\begin{proof}
%The discriminant of $C$ is 
%$$
%D = b(a^2b -4b^2 ).
%$$
%By Nagell--Lutz, $y^2\mid D$ for any point of finite order. But 
%$$
%y^2 = x(x^2 +ax+b),
%$$
%so $x\mid y^2$ and therefore $x\mid D$.  % where to go from here?
%\end{proof}
%(b) Let $p$ be a prime. Prove that the only point of finite order on the curve $C : y^2=x^3+px$ are $\OO$ and $T$.
%\begin{proof}
%See exercise 2.10.
%\end{proof}
%(c) Let $D\neq 0$ be an integer. Prove that the points of finite order on the curve $y^2 = x^3+Dx$ are as described in the following table:
%$$
%\{ P\in C(\Q) : P \textrm{ has finite order}\} \isomto
%\begin{cases}
%\Z/4\Z & D=4d^4\\
%\Z/2\Z\oplus \Z/2\Z & D = -d^2\\
%\Z/2\Z &\textrm{otherwise}
%\end{cases}.
%$$
%\begin{proof}
%
%\end{proof}
\item For primes $p$, let $C_p$ be the cubic curve $y^2=x^3+px$.\\
(a) Prove that the rank of $C_p$ is either $0$, $1$, or $2$.
\begin{proof}
Silverman and Tate define a ``useful homomorphism''
$$
\alpha : C(\Q) \longrightarrow \Q^* /\Q^{*2} 
$$
where, if $b=p$,
\begin{align*}
\alpha(\OO) &= 1 \pmod{\Q^{*2}}\\
\alpha(T) &= b \pmod{\Q^{*2}}\\
\alpha(x,y) &= x \pmod{\Q^{*2}} \quad\textrm{if }x\neq 0.
\end{align*}
In particular, one takes $\ol b = a^2 - 4b$ (where $a=0$ in this case) and considers the curve $\ol C : y^2 = x^3 + \ol bx$. From here, this homomorphism is used to derive a formula for the rank $r$ of $C$:
$$
2^r = \frac{\#\alpha(C(\Q)) \cdot\#\ol\alpha(\ol {C(\Q)})}{4}.
$$ 
One determines the orders of $\alpha(C(\Q))$ and $\ol\alpha( \ol{C(\Q)})$ by considering all of the integer factors of $b$ (and $\ol b$) modulo rational squares. For instance, if $b=4$, then this set would be $\{\pm1, \pm2\}$. For each value $\beta$ in this set, one sees if
$$
N^2 = \beta M^4 + a M^2e^2 + (b/\beta)e^4
$$
has a solution in the integers with $M\neq 0$, $\gcd(M,e) = \gcd (N, e) = \gcd (\beta,e)=1$, and, assuming $\gcd(b/\beta,M)=1$, $\gcd(b/\beta, M) = \gcd(M,N) = 1$. If it does, then it corresponds to an element in $\alpha(C(\Q))$ whose coordinates depend on $b$, $M$, $N$, and $e$ in the following way:
$$
x = \frac{bM^2}{e^2},\quad y = \frac{bMN}{e^3}.
$$ 
With this algorithm in mind, one takes $b=p$ and $\ol b = -4p$. The relevant factors of $b$ are $\pm 1$ and $\pm p$, and out of the four equations that arise, only two of them may have solutions, so $\#\alpha \le  2 $. The relevant factors of $\ol b$ are $\pm 1$, $\pm 2$, $\pm p$, and $\pm 2p$ after modding out by squares. $8$ equations arise, and all of them may have a solution. So $\#\ol\alpha( \ol{C(\Q)}) \le  8$. Taking the maximum values for each, one determines that $r=2$, so $r\le  2$. One needs to show that each $r$ occurs.
\end{proof}
%(b) If $p \equiv 7\pmod {16}$, prove that $C_p$ has rank $0$.
%\begin{proof}
%
%\end{proof}
%(c) If $p \equiv 3\pmod {16}$, prove that $C_p$ has rank either $0$ or $1$. 
%\begin{proof}
%
%\end{proof}
\item Using the method developed in Section 6, find the rank of each of the following curves.\\
(a) \qquad$y^2 = x^3+3x$.
\begin{proof}
Decomposing $b=3$, one attains $\pm 1$ and $\pm 3$ as potential solutions. $1$ corresponds to $\OO$ and $3$ corresponds to $T$. One finds when setting up the equations for $-1$ and $-3$ that $N^2<0$, so there are no more solutions. Therefore $\#\alpha(G) = 2$. Decomposing $\ol b = -12$, one attains $\pm 1$, $\pm 2$, $\pm 3$, and $\pm 6$ modulo squares.  $1$ corresponds to $\OO$ and $-12 \equiv -3\pmod{\Q^{*2}}$ corresponds to $T$. By Fermat's little theorem, the equations corresponding to $-1$, $2$, and $-6$ have no integer solutions. For example, with $\beta=2$, one has the equation
$$
N^2 = 2M^4 - 6e^4.
$$
It is enough to show that there are no solutions with $\gcd(M, 6)=1$ by the coprimality constraints. Fermat's little theorem says that $M^2 \equiv 1\pmod 3$. Taking the above equation modulo $3$, one attains $N^2 \equiv 2\pmod 3$. There are no solutions to this congruence, so this equation has no solutions. One takes a similar approach to the other mentioned case. One sees that $(N,M,e) = (2,1,1)$ is a solution corresponding to $-2$, and thus to $6$. So far, one has $4\le  \#\ol\alpha(\ol G) \le  6$. But $\#\ol\alpha(\ol G)$ must be a power of $2$, so $\#\ol\alpha(\ol G) = 4$. By the rank formula, the rank of $C$ is $1$.
\end{proof}
(b) \qquad$y^2 = x^3+5x$.
\begin{proof}
Decomposing $b=5$, one gets $\{ \pm1,\pm 5\}$ as the set of potential solutions. $1$ and $5$ correspond to $\OO$ and $T$ respectively, and $-1$ and $-5$ have no solutions. So $\# \alpha(G) = 2$. Decomposing $\ol b = -20$, one gets $\{\pm 1,\pm2,\pm5,\pm 10\}$ as the set of potential solutions. $1$ and $-20\equiv -5\pmod{\Q^{*2}}$ correspond to $\OO$ and $T$ respectively. $-1$ corresponds to the solution $(N, M, e) = (2,2,1)$. $5$ corresponds to the solution $(N, M, e) = (1,1,1)$. One can pair $2$ with $-10$ and $-2$ with $10$ as they correspond to the same equation. The equation corresponding to $2$ is 
$$
N^2 = 2M^4 -10e^4.
$$
Assuming $\gcd(M,10)=1$, Fermat's little theorem says that $M^4 \equiv 1\pmod 5$. Reducing the above equation modulo $5$, one attains $N^2 \equiv 2\pmod 5$, which has no solutions. So the above equation has no solutions. Since $\#\ol\alpha(\ol G)$ must be a power of $2$, one concludes that $\#\ol\alpha(\ol G) = 4$, and by the rank formula, the rank of $C$ is $1$.
\end{proof}
\end{enumerate}
\medskip

\bibliographystyle{amsalpha}
\bibliography{bib}
\end{document}